
Implementation of cphVB
-----------------------

In order to demonstrate our cphVB design we have implemented a basic cphVB setup. This concretization of cphVB is by no means exhaustive. The setup is targeting the NumPy library executing on a single machine with multiple CPU-cores. In this section, we will describe the implementation of each component in the cphVB setup – the Bridge, the Vector Engine Manager, and the Vector Engine. The cphVB design rules (Sec. Design) govern the interplay between the components.

Bridge
~~~~~~

The role of the Bridge is to introduce cphVB into an already existing project. In this specific case NumPy, but could just as well be ``R`` or any other language/tool that works primarily on vectorizable operations on large data objects. 

It is the responsibility of the Bridge to generate cphVB instructions on basis of the Python program that is being run. The NumPy Bridge is an extension of NumPy version 1.6. It uses hooks to divert function call where the program access cphVB enabled NumPy arrays. The hooks will translate a given function into its corresponding cphVB bytecode when possible. When it is not possible, the hooks will feed the function call back into NumPy and thereby forcing NumPy to handle the function call itself.

The Bridge operates with two address spaces for arrays: the cphVB space and the NumPy space. All arrays starts in the NumPy space as a default. The original NumPy implementation handles these arrays and all operations using them. It is possible to assign an array to the cphVB space explicitly by using an optional cphVB parameter in array creation functions such as ``empty`` and ``random``. The cphVB bridge implementation handles these arrays and all operations using them. 

In two circumstances, it is possible for an array to transfer from one address space to the other implicitly at runtime. 

 1. When an operation accesses an array in the cphVB address space but it is not possible for the bridge to translate the operation into cphVB code. In this case, the bridge will synchronize and move the data to the NumPy address space. For efficiency no data is actually copied instead the bridge uses the ``mremap`` [*]_ function to re-map the relevant memory pages. 
 2. When an operations access arrays in different address spaces the Bridge will transfer the arrays in the NumPy space to the cphVB space. Afterwards, the bridge will translate the operation into bytecode that cphVB can execute.

In order to detect direct access to arrays in the cphVB address space by the user, the original NumPy implementation, a Python library or any other external source, the bridge protects the memory of arrays that are in the cphVB address space using ``mprotect`` [*]_. Because of this memory protection, subsequently accesses to the memory will trigger a segmentation fault. The Bridge can then handle this kernel signal by transferring the array to the NumPy address space and cancel the segmentation fault. This technique makes it possible for the Bridge to support all valid Python/NumPy application since it can always fallback to the original NumPy implementation.

In order to gather greatest possible information at runtime, the Bridge will collect a batch of instructions rather than executing one instruction at a time. The Bridge will keep recording instruction until either the application reaches the end of the program or untranslatable NumPy operations forces the Bridge to move an array to the NumPy address space. When this happens, the Bridge will call the Vector Engine Manager to execute all instructions recorded in the batch.

Vector Engine Manager
~~~~~~~~~~~~~~~~~~~~~

The Vector Engine Manager (VEM) in our setup is very simple because it only has to handle one Vector Engine thus all operations go to the same Vector Engine. Still, the VEM creates and deletes arrays based on specification from the Bridge and handles all meta-data associated with arrays. 

Vector Engine
~~~~~~~~~~~~~

In order to maximize the CPU cache utilization and enables parallel execution the first stage in the VE is to form a set of instructions that enables data blocking. That is, a set of instructions where all instructions can be applied on one data block completely at a time without violating data dependencies. This set of instructions will be referred to as a kernel.

The VE will form the kernel based on the batch of instructions it receives from the VEM. The VE examines each instruction sequentially and keep adding instruction to the kernel until it reaches an instruction that is not **blockable** with the rest of the kernel. In order to be blockable with the rest of the kernel an instruction must satisfy the following two properties where :math:`A` is all instructions in the kernel and :math:`N` is the new instruction.

1. The input arrays of :math:`N` and the output array of :math:`A` do not share any data or represents precisely the same data.
2. The output array of :math:`N` and the input and output arrays of :math:`A` do not share any data or represents precisely the same data.

When the VE has formed a kernel, it is ready for execution. Since all instruction in a kernel supports data blocking the VE can simply assign one block of data to each CPU-core in the system and thus utilizing multiple CPU-cores. In order to maximize the CPU cache utilization the VE may divide the instructions into even more data blocks. The idea is to access data in chunks that fits in the CPU cache. The user, through an environment variable, manually configures the number of data blocks the VE will use.

.. [*] The function mremap() in GNU C library 2.4 and greater.
.. [*] The function mprotect() in the POSIX.1-2001 standard.

