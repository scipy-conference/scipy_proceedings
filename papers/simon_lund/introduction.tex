
Introduction
------------

Obtaining high performance from today's computing environments requires both a deep and broad working knowledge on computer architecture, communication paradigms and programming interfaces. Today's computing environments are highly heterogeneous consisting of a mixture of CPUs, GPUs, FPGAs and DSPs orchestrated in a wealth of architectures and lastly connected in numerous ways.

Utilizing this broad range of architectures manually requires programming specialists and is a very time-consuming task – time and specialization a scientific researcher typically does not have. A high-productivity language that allows rapid prototyping and still enables efficient utilization of a broad range of architectures is clearly preferable. 
There exist high-productivity language and libraries that automatically utilize parallel architectures [Kri10]_ [Dav04]_ [New11]_. They are however still few in numbers and have one problem in common. They are closely coupled to both the front-end, i.e. programming language and IDE, and the back-end, i.e. computing device, which makes them interesting only to the few using the exact combination of front and back-end.

A tight coupling between front-end technology and back-end presents another problem; the usefulness of the developed program expires as soon as the back-end does. With the rapid development of hardware architectures the time spend on implementing optimized programs for specific hardware, is lost as soon as the hardware product expires.
 
In this paper, we present a novel approach to the problem of closing the gap between high-productivity languages and parallel architectures, which allows a high degree of modularity and reusability. The approach involves creating a framework, cphVB [*]_ (Copenhagen Vector Bytecode). cphVB defines a clear and easy to understand intermediate bytecode language and provides a runtime environment for executing the bytecode. cphVB also contains a protocol to govern the safe, and efficient exchange, creation, and destruction of model data.

cphVB provides a retargetable framework in which the user can write programs utilizing whichever cphVB supported programming interface they prefer and run the program on their own workstation while doing prototyping, such as testing correctness and functionality of their programs. Users can then deploy exactly the same program in a more powerful execution environment without changing a single line of code and thus effectively solve greater problem sets.

The rest of the paper is organized as follows. In Section `Programming Model`. we describe the programming model supported by cphVB. The section following gives a brief description of `Numerical Python`, which is the first programming interface that fully supports cphVB. Sections `Design` and `Implementation` cover the overall cphVB design and an implementation of it. In Section `Performance Study`, we conduct an evaluation of the implementation. Finally, in Section `Future Work` and `Conclusion` we discuss future work and conclude.

.. [*] Open Source Project - Website: http://cphvb.bitbucket.org.

