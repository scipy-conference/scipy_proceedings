
Performance Study
-----------------

.. table:: ASUS P31SD. :label:`tab:specs`

   +------------------------------+----------------------+
   | Processor                    | Intel Core i5-2510M  |
   +------------------------------+----------------------+
   | Clock                        | 2.3 GHz              |     	
   +------------------------------+----------------------+
   | Private L1 Data Cache        | 128 KB               |
   +------------------------------+----------------------+
   | Private L2 Data Cache        | 512 KB               |
   +------------------------------+----------------------+
   | Shared L3 Cache              | 3072 KB              |
   +------------------------------+----------------------+
   | Memory Bandwidth             | 21.3 GB/s            |
   +------------------------------+----------------------+
   | Memory                       | 4GB DDR3-1333        |
   +------------------------------+----------------------+
   | Compiler                     | GCC 4.6.3            |
   +------------------------------+----------------------+

In order to demonstrate the performance of our initial cphVB implementation and thereby the potential of the cphVB design, we will conduct some performance benchmarks using NumPy [*]_. We execute the benchmark applications on ASUS P31SD with an Intel Core i5-2410M processor (Table :ref:`tab:specs`). 

The experiments used the three vector engines: `simple`, `score` and `mcore` and for each execution we calculate the relative speedup of cphVB compared to NumPy. We perform strong scaling experiments, in which the problem size is constant though all the executions. For each experiment, we find the block size that results in best performance and we calculate the result of each experiment using the average of three executions.

The benchmark consists of the following Python/NumPy applications. All are pure Python applications that make use of NumPy and none uses any external libraries.

 - **Jacobi Solver** An implementation of an iterative jacobi solver with fixed iterations instead of numerical convergence. (Fig. :ref:`benchmark:jacobi`). 

 - **kNN** A naive implementation of a k Nearest Neighbor search (Fig. :ref:`benchmark:knn`). 

 - **Shallow Water** A simulation that simulates a system governed by the shallow water equations. It is a translation of a MATLAB application by Burkardt [Bur10]_ (Fig. :ref:`benchmark:swater`). 

 - **Synthetic Stencil** A synthetic stencil simulation the code relies heavily on the slicing operations of NumPy. (Fig. :ref:`benchmark:stencil`).


Discussion
~~~~~~~~~~

The jacobi solver shows an efficient utilization of data-blocking to an extent competing with using multiple processors. The `score` engine achieves a 1.42x speedup in comparison to NumPy (:math:`3.98sec` to :math:`2.8sec`).

On the other hand, our naive implementation of the k Nearest Neighbor search is not an embarrassingly parallel problem. However, it has a time complexity of :math:`O(n^2)` when the number of elements and the size of the query set is :math:`n`, thus the problem should be scalable. The result of our experiment is also promising – with a performance speedup of of 3.57x (:math:`5.40sec` to :math:`1.51sec`) even with the two single-core engines and a speed-up of nearly 6.8x (:math:`5.40sec` to :math:`0.79`)  with the multi-core engine.

The Shallow Water simulation only has a time complexity of :math:`O(n)` thus it is the most memory intensive application in our benchmark. Still, cphVB manages to achieve a performance speedup of 1.52x (:math:`7.86sec` to :math:`5.17sec`) due to memory-allocation optimization and 2.98x (:math:`7.86sec` to :math:`2.63sec`) using the multi-core engine. 

Finally, the synthetic stencil has an almost identical performance pattern as the shallow water benchmark the `score` engine does however give slightly better results than the `simple` engine. Score achieves a speedup of 1.6x (:math:`6.60sec` to :math:`4.09sec`) and the `mcore` engine achieves a speedup of 3.04x (:math:`6.60sec` to :math:`2.17sec`).

It is promising to observe that even most basic vector engine (`simple`) shows a speedup and in none of our benchmarks a slowdown. This leads to the promising conclusion that the memory optimizations implemented outweigh the cost of using cphVB. Adding the potential of speedup due to data-blocking motivates studying further optimizations in addition to thread-level-parallelization.
The `mcore` engine does provide speedups, the speedup does however not scale with the number of cores. This result is however expected as the benchmarks are memory-intensive and the memory subsystem is therefore the bottleneck and not the number of computational cores available.

.. figure:: ../../../benchpress/jacobi_fixed_speedup.pdf

   Relative speedup of the Jacobi Method. The job consists of a vector with :math:`7168x7168` elements using four iterations. :label:`benchmark:jacobi`

.. figure:: ../../../benchpress/knn_speedup.pdf

   Relative speedup of the k Nearest Neighbor search. The job consists of 10.000 elements and the query set also consists of 1K elements. :label:`benchmark:knn`

.. figure:: ../../../benchpress/swater_speedup.pdf

   Relative speedup of the Shallow Water Equation. The job consists of 10.000 grid points that simulate 120 time steps. :label:`benchmark:swater`

.. figure:: ../../../benchpress/stencil_speedup.pdf

   Relative speedup of the synthetic stencil code. The job consists of vector with :math:`10240x1024` elements that simulate 10 time steps. :label:`benchmark:stencil`

.. [*] NumPy version 1.6.1.
