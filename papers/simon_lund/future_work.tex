
Future Work
-----------

The future goals of cphVB involves improvement in two major areas; expanding support and improving performance. Work has started on a CIL-bridge which will leverage the use of cphVB to every CIL based programming language which among others include: C\#, F\#, Visual C++ and VB.NET. Another project in current progress within the area of support is a C++ bridge providing a library-like interface to cphVB using operator overloading and templates to provide a high-level interface in C++.

To improve both support and performance, work is in progress on a vector engine targeting OpenCL compatible hardware, mainly focusing on using GPU-resources to improve performance. Additionally the support for program execution using distributed memory is on progress. This functionality will be added to cphVB in the form a vector engine manager.

In terms of pure performance enhancement, cphVB will introduce JIT compilation in order to improve memory intensive applications. The current vector engine for multi-cores CPUs uses data blocking to improve cache utilization but as our experiments show then the memory intensive applications still suffer from the von Neumann bottleneck [Bac78]_. By JIT compile the instruction kernels, it is possible to improve cache utilization drastically.
