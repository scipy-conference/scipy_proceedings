
Design of cphVB
---------------

The key contribution in this paper is a framework, cphVB, that support a vector oriented programming model. The idea of cphVB is to provide the mechanics to seamlessly couple a programming language or library with an architecture-specific implementation of vectorized operations.

cphVB consists of a number of components that communicate using a simple protocol. Components are allowed to be architecture-specific but they are all interchangeable since all uses the same communication protocol. The idea is to make it possible to combine components in a setup that perfectly match a specific execution environment. cphVB consist of the following components:

Programming Interface
  The programming language or library exposed to the user. cphVB was initially meant as a computational back-end for the Python library NumPy, but we have generalized cphVB to potential support all kinds of languages and libraries. Still, cphVB has design decisions that are influenced by NumPy and its representation of vectors/matrices.

Bridge
  The role of the Bridge is to integrate cphVB into existing languages and libraries. The Bridge generates the cphVB bytecode that corresponds to the user-code.

Vector Engine
  The Vector Engine is the architecture-specific implementation that executes cphVB bytecode. 

Vector Engine Manager
  The Vector Engine Manager manages data location and ownership of vectors. It also manages the distribution of computing jobs between potentially several Vector Engines, hence the name.

An overview of the design can be seen in Figure :ref:`fig-cphvb-design`.

.. figure:: ../../../gfx/cphvb.pdf

   cphVB design idea. :label:`fig-cphvb-design`

Configuration
~~~~~~~~~~~~~

To make cphVB as flexible a framework as possible, we manage the setup of all the components at runtime through a configuration file. The idea is that the user can change the setup of components simply by editing the configuration file before executing the user application. Additionally, the user only has to change the configuration file in order to run the application on different systems with different computational resources. The configuration file uses the ini syntax, an example is provided below.


.. code-block:: c

   # Root of the setup
   [setup]
   bridge = numpy
   debug = true

   # Bridge for NumPy
   [numpy]
   type = bridge
   children = node

   # Vector Engine Manager for a single machine
   [node]
   type = vem
   impl = libcphvb_vem_node.so
   children = mcore

   # Vector Engine using TLP on shared memory
   [mcore]
   type = ve
   impl = libcphvb_ve_mcore.so


This example configuration provides a setup for utilizing a shared memory machine with thread-level-parallelism (TLP) on one machine by instructing the vector engine manager to use a single multi-core TLP engine.

Bytecode
~~~~~~~~~

The central part of the communication between all the components in cphVB is vector bytecode. The goal with the bytecode language is to be able to express operations on multidimensional vectors. Taking inspiration from single instruction, multiple data (SIMD) instructions but adding structure to the data. This, of course, fits very well with the array operations in NumPy but is not bound nor limited to these.

We would like the bytecode to be a concept that is easy to explain and understand. It should have a simple design that is easy to implement. It should be easy and inexpensive to generate and decode. To fulfill these goals we chose a design that conceptually is an assembly language where the operands are multidimensional vectors. Furthermore, to simplify the design the assembly language should have a one-to-one mapping between instruction mnemonics and opcodes.

In the basic form, the bytecode instructions are primitive operations on data, e.g. addition, subtraction, multiplication, division, square root etc. As an example, let us look at addition. Conceptually it has the form::

    add $d, $a, $b

Where ``add`` is the opcode for addition. After execution ``$d`` will contain the sum of ``$a`` and ``$b``.

The requirement is straightforward: we need an opcode. The opcode will explicitly identify the operation to perform. Additionally the opcode will implicitly define the number of operands. Finally, we need some sort of symbolic identifiers for the operands. Keep in mind that the operands will be multidimensional arrays.

Interface
~~~~~~~~~

The Vector Engine and the Vector Engine Manager exposes simple API that consists of the following functions: initialization, finalization, registration of a user-defined operation and execution of a list of bytecodes. Furthermore, the Vector Engine Manager exposes a function to define new arrays.

Bridge
~~~~~~

The Bridge is the **bridge** between the programming interface, e.g. Python/NumPy, and the Vector Engine Manager. The Bridge is the only component that is specifically implemented for the programming interface. In order to add cphVB support to a new language or library, one only has to implement the bridge component. It generates bytecode based on programming interface and sends them to the Vector Engine Manager.

Vector Engine Manager
~~~~~~~~~~~~~~~~~~~~~

Instead of allowing the front-end to communicate directly with the Vector Engine, we introduce a Vector Engine Manager (VEM) into the design. It is the responsibility of the VEM to manage data ownership and distribute bytecode instructions to several Vector Engines. It is also the ideal place to implement code optimization, which will benefit all Vector Engines.

To facilitate late allocation, and early release of resources, the VEM handles instantiation and destruction of arrays. At array creation only the meta data is actually created. Often arrays are created with structured data (e.g. random, constants), with no data at all (e.g. empty), or as a result of calculation. In any case it saves, potentially several, memory copies to delay the actual memory allocation. Typically, array data will exist on the computing device exclusively.

In order to minimize data copying we introduce a data ownership scheme. It keeps track of which components in cphVB that needs to access a given array. The goal is to allow the system to have several copies of the same data while ensuring that they are in synchronization. We base the data ownership scheme on two instructions, **sync** and **discard**:

Sync 
  is issued by the bridge to request read access to a data object. This means that when acknowledging a **sync** request, the copy existing in shared memory needs to be the most resent copy.

Discard
  is used to signal that the copy in shared memory has been updated and all other copies are now invalid. Normally used by the bridge to upgrading a read access to a write access.

The cphVB components follow the following four rules when implementing the data ownership scheme:

1. The Bridge will always ask the Vector Engine Manager for access to a given data object. It will send a **sync** request for read access, followed by a **release** request for write access. The Bridge will not keep track of ownership itself.

2. A Vector Engine can assume that it has write access to all of the output parameters that are referenced in the instructions it receives. Likewise, it can assume read access on all input parameters.

3. A Vector Engine is free to manage its own copies of arrays and implement its own scheme to minimize data copying. It just needs to copy modified data back to share memory when receiving a **sync** instruction and delete all local copies when receiving a **discard** instruction.

4. The Vector Engine Manager keeps track of array ownership for all its children. The owner of an array has full (i.e. write) access. When the parent component of the Vector Engine Manager, normally the Bridge, request access to an array, the Vector Engine Manager will forward the request to the relevant child component. The Vector Engine Manager never accesses the array itself.

Additionally, the Vector Engine Manager needs the capability to handle multiple children components. In order to maximize parallelism the Vector Engine Manager can distribute workload and array data between its children components.

Vector Engine
~~~~~~~~~~~~~

Though the Vector Engine is the most complex component of cphVB, it has a very simple and a clearly defined role. It has to execute all instructions it receives in a manner that obey the serialization dependencies between instructions. Finally, it has to ensure that the rest of the system has access to the results as governed by the rules of the **sync**, **release**, and **discard** instructions.

