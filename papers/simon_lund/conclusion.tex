
Conclusion
----------

The vector oriented programming model used in cphVB provides a framework for high-performance and high-productivity. It enables the end-user to execute vectorized applications on a broad range of hardware architectures efficiently without any hardware specific knowledge. Furthermore, the cphVB design supports scalable architectures such as clusters and supercomputers. It is even possible to combine architectures in order to exploit hybrid programming where multiple levels of parallelism exist. The authors in [Kri11]_ demonstrate that combining shared memory and distributed memory parallelism through hybrid programming is essential in order to utilize the Blue Gene/P architecture fully.

In a case study, we demonstrate the design of cphVB by implementing a front-end for Python/NumPy that targets multi-core CPUs in a shared memory environment. The implementation executes vectorized applications in parallel without any user intervention. Thus showing that it is possible to retain the high abstraction level of Python/NumPy while fully utilizing the underlying hardware. Furthermore, the implementation demonstrates scalable performance – a k-nearest neighbor search purely written in Python/NumPy obtains a speedup of more than five compared to a native execution.

Future work will further test the cphVB design model as new front-end technologies and heterogeneous architectures are supported. 
