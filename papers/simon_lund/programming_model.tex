
Target Programming Model
------------------------

To hide the complexities of obtaining high-performance from a heterogeneous environment any given system must provide a meaningful high-level abstraction. This can be realized in the form of domain specific languages, embedded languages, language extensions, libraries, APIs etc. Such an abstraction serves two purposes: 1) It must provide meaning for the end-user such that the goal of high-productivity can be met with satisfaction. 2) It must provide an abstraction that consists of a sufficient amount of information for the system to optimize its utilization.

cphVB is not biased towards any specific choice of abstraction or front-end technology as long as it is compatible with a vector oriented programming model. This provides means to use cphVB in functional programming languages, provide a front-end with a strict mathematic notation such as APL [Apl00]_ or a more relaxed syntax such as MATLAB.

The vector oriented programming model encourages expressing programs in the form of high-level array operations, e.g. by expressing the addition of two arrays using one high-level function instead of computing each element individually. The NumPy application in the code example above figure :ref:`fig-stencil-expr` is a good example of using the vector oriented programming model.
